\documentclass[a4paper,oneside,12pt]{article}
\usepackage[slovene]{babel}
\usepackage[utf8]{inputenc}
\usepackage[T1]{fontenc}
\usepackage{url}
\usepackage{graphicx}
\usepackage[usenames]{color}
\usepackage[reqno]{amsmath}
\usepackage{amssymb,amsthm}
\usepackage{enumerate}
\usepackage{array}
\usepackage[colorlinks=true,
  linkcolor=black, anchorcolor=black, citecolor=black, filecolor=black,
  menucolor=black, runcolor=black, urlcolor=black, pdfencoding=unicode,
  bookmarks=true, bookmarksopenlevel=2
]{hyperref}
\usepackage[
  paper=a4paper,
  top=2.5cm,
  bottom=2.5cm,
  textwidth=15cm,
]{geometry}

\usepackage{icomma}
\usepackage{units}
\usepackage{minted}
\usepackage{nameref}

\def\R{\mathbb{R}}
\def\N{\mathbb{N}}
\def\Z{\mathbb{Z}}
\def\C{\mathbb{C}}
\def\Q{\mathbb{Q}}

\newenvironment{description*}%
{
\begin{description}
\setlength{\itemsep}{0pt}
\setlength{\parskip}{2pt}
}
{\end{description}}

\newcommand{\mytitle}{Codebook}
\title{\mytitle}
\author{Jure Slak}
\date{\today}
\hypersetup{pdftitle={\mytitle}}
\hypersetup{pdfauthor={Jure Slak}}
\hypersetup{pdfsubject={}}

\newmintedfile[cppsource]{c++}{linenos=true, mathescape, xleftmargin=0.7cm,
                               fontsize=\scriptsize,baselinestretch=0.9,firstline=3}
\newmintedfile[pysource]{python}{linenos=true, mathescape, xleftmargin=0.7cm,
                                 fontsize=\scriptsize,baselinestretch=0.9}

\newcommand{\ull}{\texttt{unsigned long long}}

\begin{document}

\thispagestyle{empty}

\vspace*{\fill}
\begin{center}
  \scalebox{6}{\texttt{Codebook}}\\[6ex]
  \scalebox{2}{Pitoni\textbf{++}}\\[4ex]
  Žiga Gosar, Maks Kolman, Jure Slak
  \vfill
  verzija: \today
\end{center}

\newpage

\tableofcontents

\newpage

\section{Teorija števil}
\label{sec:ts}

\subsection{Evklidov algoritem}
\label{sec:ts:evk}

\begin{description*}
  \item[Vhod:] $a, b \in \Z$
  \item[Izhod:] Največji skupni delitelj $a$ in $b$. Za pozitivna števila je
    pozitiven, če je eno število 0, je rezultat drugo število, pri negativnih je
    predznak odvisen od števila iteracij.
  \item[Časovna zahtevnost:] $O(\log(\max\{a, b\}))$
  \item[Prostorska zahtevnost:] $O(1)$
\end{description*}
\cppsource{implementacija/ts/gcd.cpp}

\subsection{Razširjen Evklidov algoritem}
\label{sec:ts:extevk}
\begin{description*}
  \item[Vhod:] $a, b \in \Z,$. Števili $retx$, $rety$ sta parametra samo za vračanje vrednosti.
  \item[Izhod:] Števila $x, y, d$, pri čemer $d = \gcd(a, b)$, ki rešijo
    Diofantsko enačbo $ax + by = d$. V posebnem primeru, da je $b$ tuj $a$, je
    $x$ inverz števila $a$ v multiplikativni grupi $Z_b^\ast$.
  \item[Časovna zahtevnost:] $O(\log(\max\{a, b\}))$
  \item[Prostorska zahtevnost:] $O(1)$
  \item[Testiranje na terenu:] UVa 756
\end{description*}
\cppsource{implementacija/ts/extended_gcd.cpp}

\subsection{Kitajski izrek o ostankih}
\label{sec:ts:mod}
\begin{description*}
  \item[Vhod:] Sistem $n$ kongruenc $x \equiv a_i \pmod{m_i}$, $m_i$ so paroma tuji.
  \item[Izhod:] Število $x$, ki reši ta sistem dobimo po formuli
    \[ x =
      \left[\sum_{i=1}^na_i\frac{M}{m_i}\left[\left(\frac{M}{mi}\right)^{-1}\right]_{m_i}\right]_M,
      \qquad M = \prod_{i=1}^nm_i,
    \]
    kjer $[x^{-1}]_m$ označuje inverz $x$ po modulu $m$. Vrnjeni $x$ je med 0 in $M$.
  \item[Časovna zahtevnost:] $O(n \log(\max\{m_i, a_i\}))$
  \item[Prostorska zahtevnost:] $O(n)$
  \item[Potrebuje:] \nameref{sec:ts:evk} (str.~\pageref{sec:ts:evk})
  \item[Testiranje na terenu:] UVa 756
  \item[Opomba:] Pogosto potrebujemo \ull{} namesto \texttt{int}.
\end{description*}
\cppsource{implementacija/ts/chinese_reminder_theorem.cpp}

\subsection{Hitro potenciranje}
\label{sec:fastpow}
\begin{description*}
  \item[Vhod:] Število $g$ iz splošne grupe in $n \in \N_0$.
  \item[Izhod:] Število $g^n$.
  \item[Časovna zahtevnost:] $O(\log(n))$
  \item[Prostorska zahtevnost:] $O(1)$
  \item[Testiranje na terenu:] \url{http://putka.upm.si/tasks/2010/2010_3kolo/nicle}
\end{description*}
\cppsource{implementacija/ts/fast_power.cpp}

\end{document}
% vim: spell spelllang=sl
% vim: foldlevel=99
