\documentclass[a4paper,oneside,12pt]{article}
\usepackage[slovene]{babel}
\usepackage[utf8]{inputenc}
\usepackage[T1]{fontenc}
\usepackage{url}
\usepackage{graphicx}
\usepackage[usenames]{color}
\usepackage[reqno]{amsmath}
\usepackage{amssymb,amsthm,mathabx}
\usepackage{enumerate}
\usepackage{array}
\usepackage[bookmarks, bookmarksopen, bookmarksdepth=3, colorlinks=true,
  linkcolor=black, anchorcolor=black, citecolor=black, filecolor=black,
  menucolor=black, runcolor=black, urlcolor=black, pdfencoding=unicode
]{hyperref}
\usepackage[
  paper=a4paper,
  top=2.5cm,
  bottom=2.5cm,
  textwidth=15cm,
]{geometry}

\usepackage{icomma}
\usepackage{units}
\usepackage{minted}
\usepackage{nameref}

\def\R{\mathbb{R}}
\def\N{\mathbb{N}}
\def\Z{\mathbb{Z}}
\def\C{\mathbb{C}}
\def\Q{\mathbb{Q}}

\newenvironment{itemize*}%
{
\vspace{-6pt}
\begin{itemize}
\setlength{\itemsep}{0pt}
\setlength{\parskip}{2pt}
}
{\end{itemize}}

\newenvironment{description*}%
{
\begin{description}
\setlength{\itemsep}{0pt}
\setlength{\parskip}{2pt}
}
{\end{description}}

\newcommand{\mytitle}{Codebook}
\title{\mytitle}
\author{Jure Slak}
\date{\today}
\hypersetup{pdftitle={\mytitle}}
\hypersetup{pdfauthor={Jure Slak}}
\hypersetup{pdfsubject={}}

\newmintedfile[cppsource]{c++}{linenos=true, mathescape, xleftmargin=0.7cm,
                               fontsize=\scriptsize,baselinestretch=0.9,firstline=3}
\newmintedfile[pysource]{python}{linenos=true, mathescape, xleftmargin=0.7cm,
                                 fontsize=\scriptsize,baselinestretch=0.9}

\newcommand{\cpp}[1]{\cppsource{implementacija/#1}}
\newcommand{\ull}{\texttt{unsigned long long}}

\begin{document}

\thispagestyle{empty}

\vspace*{\fill}
\begin{center}
  \scalebox{6}{\texttt{Codebook}}\\[6ex]
  \scalebox{2}{Pitoni\textbf{++}}\\[4ex]
  Žiga Gosar, Maks Kolman, Jure Slak
  \vfill
  verzija: \today
\end{center}

\newpage

\tableofcontents

\newpage

\setcounter{section}{-1}
\section{Uvod}
Napotki zame:
\begin{itemize*}
  \item podrobno in pozorno preberi navodila
  \item pazi na \texttt{double} in \texttt{ull}, oceni velikost rezultata
\end{itemize*}

\section{Grafi}
\label{sec:grafi}

\subsection{Topološko sortiranje}
\begin{description*}
  \item[Vhod:] Število vozlišč $n$ in število povezav $m$ ter seznam povezav $E$
    oblike $u \to v$ dolžine $m$. Usmerjen graf $G$ je tako sestavljen iz vozlišč z oznakami
    0 do $n-1$ in povezavami iz $E$. $G$ ne sme imeti zank, če pa jih ima, se
    jih lahko brez škode odstrani.
  \item[Izhod:] Topološka ureditev usmerjenega grafa $G$, to je seznam vozlišč v takem
    vrstnem redu, da nobena povezava ne kaže nazaj. Če je vrnjeni seznam krajši
    od $n$, potem ima $G$ cikle.
  \item[Časovna zahtevnost:] $O(V + E)$
  \item[Prostorska zahtevnost:] $O(V)$
  \item[Testiranje na terenu:] UVa 10305
\end{description*}
\cpp{grafi/topological_sort.cpp}

\subsection{Mostovi in prerezna vozlišča grafa}
\label{sec:bridge}
\begin{description*}
  \item[Vhod:] Število vozlišč $n$ in število povezav $m$ ter seznam povezav $E$
    oblike $u \to v$ dolžine $m$. Neusmerjen graf $G$ je tako sestavljen iz vozlišč z oznakami
    0 do $n-1$ in povezavami iz $E$.
  \item[Izhod:] Seznam prereznih vozlišč: točk, pri katerih, če jih
    odstranimo, graf razpade na dve komponenti in seznam mostov grafa $G$: povezav, pri
    katerih, če jih odstranimo, graf razpade na dve komponenti.
  \item[Časovna zahtevnost:] $O(V + E)$
  \item[Prostorska zahtevnost:] $O(V + E)$
  \item[Testiranje na terenu:] UVa 315
\end{description*}
\cpp{grafi/articulation_points_and_bridges.cpp}

\subsection{Močno povezane komponente}
\label{sec:scc}
\begin{description*}
  \item[Vhod:] Seznam sosednosti s težami povezav.
  \item[Izhod:] Seznam povezanih komponent grafa v obratni topološki ureditvi in
    kvocientni graf, to je DAG, ki ga dobimo iz grafa, če njegove komponente
    stisnemo v točke. Morebitnih več povezav med dvema komponentama seštejemo.
  \item[Časovna zahtevnost:] $O(V + E)$
  \item[Prostorska zahtevnost:] $O(V + E)$
  \item[Testiranje na terenu:]
    \url{http://putka.upm.si/tasks/2012/2012_3kolo/zakladi}
\end{description*}
\cpp{grafi/strongly_connected_components.cpp}

\subsection{Največje prirejanje in najmanjše pokritje}
\label{sec:mbm}
v angleščini \emph{Maximum cardinality bipartite matching} (če bi dodali še
kakšno povezavo bi se dve stikali) in \emph{minimum vertex cover} (če bi vzeli
še kakšno točko stran, bi bila neka povezava brez pobarvane točke na obeh
koncih).
\begin{description*}
  \item[Vhod:] Dvodelen neutežen graf, dan s seznamom sosedov. Prvih \texttt{left} vozlišč je na
    levi strani.
  \item[Izhod:] Število povezav v $MCBM$ = število točk v $MVC$, prvi $MVC$ vrne tudi neko minimalno
    pokritje. Velja tudi $MIS = V - MCBM$, $MIS$ pomeni \emph{maximum
    independent set}.
  \item[Časovna zahtevnost:] $O(VE)$
  \item[Prostorska zahtevnost:] $O(V + E)$
  \item[Testiranje na terenu:] UVa 11138
\end{description*}
\cpp{grafi/bipartite_matching.cpp}

\section{Teorija števil}
\label{sec:ts}

\subsection{Evklidov algoritem}
\label{sec:ts:evk}

\begin{description*}
  \item[Vhod:] $a, b \in \Z$
  \item[Izhod:] Največji skupni delitelj $a$ in $b$. Za pozitivna števila je
    pozitiven, če je eno število 0, je rezultat drugo število, pri negativnih je
    predznak odvisen od števila iteracij.
  \item[Časovna zahtevnost:] $O(\log(a) + \log(b))$
  \item[Prostorska zahtevnost:] $O(1)$
\end{description*}
\cpp{ts/gcd.cpp}

\subsection{Razširjen Evklidov algoritem}
\label{sec:ts:extevk}
\begin{description*}
  \item[Vhod:] $a, b \in \Z$. Števili $retx$, $rety$ sta parametra samo za vračanje vrednosti.
  \item[Izhod:] Števila $x, y, d$, pri čemer $d = \gcd(a, b)$, ki rešijo
    Diofantsko enačbo $ax + by = d$. V posebnem primeru, da je $b$ tuj $a$, je
    $x$ inverz števila $a$ v multiplikativni grupi $Z_b^\ast$.
  \item[Časovna zahtevnost:] $O(\log(a) + \log(b))$
  \item[Prostorska zahtevnost:] $O(1)$
  \item[Testiranje na terenu:] UVa 756
\end{description*}
\cpp{ts/extended_gcd.cpp}

\subsection{Kitajski izrek o ostankih}
\label{sec:ts:mod}
\begin{description*}
  \item[Vhod:] Sistem $n$ kongruenc $x \equiv a_i \pmod{m_i}$, $m_i$ so paroma tuji.
  \item[Izhod:] Število $x$, ki reši ta sistem dobimo po formuli
    \[ x =
      \left[\sum_{i=1}^na_i\frac{M}{m_i}\left[\left(\frac{M}{mi}\right)^{-1}\right]_{m_i}\right]_M,
      \qquad M = \prod_{i=1}^nm_i,
    \]
    kjer $[x^{-1}]_m$ označuje inverz $x$ po modulu $m$. Vrnjeni $x$ je med 0 in $M$.
  \item[Časovna zahtevnost:] $O(n \log(\max\{m_i, a_i\}))$
  \item[Prostorska zahtevnost:] $O(n)$
  \item[Potrebuje:] \nameref{sec:ts:evk} (str.~\pageref{sec:ts:evk})
  \item[Testiranje na terenu:] UVa 756
  \item[Opomba:] Pogosto potrebujemo \ull{} namesto \texttt{int}.
\end{description*}
\cpp{ts/chinese_reminder_theorem.cpp}

\subsection{Hitro potenciranje}
\label{sec:fastpow}
\begin{description*}
  \item[Vhod:] Število $g$ iz splošne grupe in $n \in \N_0$.
  \item[Izhod:] Število $g^n$.
  \item[Časovna zahtevnost:] $O(\log(n))$
  \item[Prostorska zahtevnost:] $O(1)$
  \item[Testiranje na terenu:] \url{http://putka.upm.si/tasks/2010/2010_3kolo/nicle}
\end{description*}
\cpp{ts/fast_power.cpp}

\subsection{Številski sestavi}
\label{sec:sestavi}
\begin{description*}
  \item[Vhod:] Število $n \in \N_0$ ali $\frac{p}{q} \in Q$ ter $b \in [2,
      \infty) \cap \N.$
  \item[Izhod:] Število $n$ ali $\frac{p}{q}$ predstavljeno v izbranem sestavu
    z izbranimi števkami in označeno periodo.
  \item[Časovna zahtevnost:] $O(\log(n))$ ali $O(q\log(q))$
  \item[Prostorska zahtevnost:] $O(n)$ ali $O(q)$
  \item[Testiranje na terenu:]
    \url{http://putka.upm.si/tasks/2010/2010_finale/ulomki}
  \item[Opomba:] Zgornja meja za bazo $b$ je dolžina niza
    \verb|STEVILSKI_SESTAVI_ZNAKI|.
\end{description*}
\cpp{ts/stevilski_sestavi.cpp}

\subsection{Eulerjeva funkcija $\phi$}
\label{sec:phi}
\begin{description*}
  \item[Vhod:] Število $n \in \N$.
  \item[Izhod:] Število $\phi(n)$, to je število števil manjših ali enakih $n$ in tujih $n$.
    Direktna formula:
    \[ \phi(n) = n\cdot \prod_{p \divides n}(1-\frac{1}{p}) \]
  \item[Časovna zahtevnost:] $O(\sqrt{n})$
  \item[Prostorska zahtevnost:] $O(1)$
  \item[Testiranje na terenu:]
    \url{https://projecteuler.net/problem=69}
\end{description*}
\cpp{ts/euler_phi.cpp}

\subsection{Eratostenovo rešeto}
\label{sec:primes}
\begin{description*}
  \item[Vhod:] Število $n \in \N$.
  \item[Izhod:] Seznam praštevil manjših od $n$ in seznam, kjer je za vsako
    število manjše od $n$ notri njegov najmanjši praštevilski delitelj. To se
    lahko uporablja za faktorizacijo števil in testiranje praštevilskosti.
  \item[Časovna zahtevnost:] $O(n\log(n))$
  \item[Prostorska zahtevnost:] $O(n)$
  \item[Testiranje na terenu:] UVa 10394
\end{description*}
\cpp{ts/eratosthenes_sieve.cpp}

\section{Geometrija}
\label{sec:geom}
Zaenkrat obravnavamo samo ravninsko geometrijo.
Točke predstavimo kot kompleksna števila. Daljice predstavimo z začetno in
končno točko. Premice s koeficienti v enačbi $ax + by = c$. Premico lahko
konstruiramo iz dveh točk in po želji hranimo točko in smerni vektor.
Pravokotnike predstavimo z spodnjim levim in zgornjim desnim ogliščem.
Večkotnike predstavimo s seznamom točk, kot si sledijo, prve točke ne
ponavljamo. Tip \texttt{ITYPE} predstavlja različne vrste presečišč ali
vsebovanosti: \texttt{OK} pomeni, da se lepo seka oz.\ je točka v notranjosti.
\texttt{NO} pomeni, da se ne seka oz.\ da točna ni vsebovana, \texttt{EQ} pa
pomeni, da se premici prekrivata, daljici sekata v krajišču ali se pokrivata,
oz.\ da je točka na robu.

\subsection{Osnove}
\label{sec:basic}
Funkcije:
\begin{itemize*}
  \item skalarni in vektorski produkt
  \item pravokotni vektor in polarni kot
  \item ploščina trikotnika in enostavnega mnogokotnika
  \item razred za premice
  \item razdalja do premice, daljice, po sferi
  \item vsebovanost v trikotniku, pravokotniku, enostavnem mnogokotniku
  \item presek dveh premic, premice in daljice in dveh daljic
  \item konstrukcije krogov iz treh točk, iz dveh točk in radija
\end{itemize*}
\begin{description*}
  \item[Vhod:] Pri argumentih funkcij.
  \item[Izhod:] Pri argumentih funkcij.
  \item[Časovna zahtevnost:] $O(\text{št.\ točk})$
  \item[Prostorska zahtevnost:] $O(\text{št.\ točk})$
  \item[Testiranje na terenu:] Bolj tako, ima pa obsežne unit teste\dots
\end{description*}
\label{sec:geom-osnove}
\cpp{geom/basics.cpp}
.
\subsection{Konveksna ovojnica}
\label{sec:convex-hull}
\begin{description*}
  \item[Vhod:] Seznam $n$ točk.
  \item[Izhod:] Najkrajši seznam $h$ točk, ki napenjajo konveksno ovojnico,
    urejen naraščajoče po kotu glede na spodnjo levo točko.
  \item[Časovna zahtevnost:] $O(n\log n)$, zaradi sortiranja
  \item[Prostorska zahtevnost:] $O(n)$
  \item[Potrebuje:] Vektorski produkt, str.~\pageref{sec:basic}.
  \item[Testiranje na terenu:] UVa 681
\end{description*}
\cpp{geom/convex_hull.cpp}

\subsection{Ploščina unije pravokotnikov}
\begin{description*}
  \item[Vhod:] Seznam $n$ pravokotnikov $P_i$ danih s spodnjo levo in zgornjo desno
    točko.
  \item[Izhod:] Ploščina unije danih pravokotnikov.
  \item[Časovna zahtevnost:] $O(n\log n)$
  \item[Prostorska zahtevnost:] $O(n)$
  \item[Testiranje na terenu:] \url{http://putka.upm.si/competitions/upm2013-2/kolaz}
\end{description*}
\cpp{geom/rectangle_union_area.cpp}

\end{document}
% vim: spell spelllang=sl
% vim: foldlevel=99
